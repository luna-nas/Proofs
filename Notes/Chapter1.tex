\documentclass{article}

\author{Luna S}
\title{Proofs - A Long-Form Mathematics Textbook - Chapter 1}

\usepackage{amsthm}

\begin{document}

\maketitle
\tableofcontents

\section{Intro}
Three Phases of Mathematics:
\begin{enumerate}
    \item Finding an answer to a problem using previously established results
    \item Finding why previously establihsed results work
    \item Making your own results for new problems
\end{enumerate}

\section{Chessboard Problems}

\begin{itemize}
    \item You can use 32 dominos of size 2x1 to cover an 8x8 chessboard perfectly, and this can happen in multiple different ways.
    \item Mathematics runs on definitions, so define a perfect cover of an $m x n$ board to be an arrangement of dominoes on the board with no squares left uncovered, no dominoes stacking, and no dominoes hanging off the edge
    \item Proofs require a proposition(something which is true and requires proof), so a proposition could be that there exists a perfect cover of an 8x8 chessboard.
    \item Proofs have key words that show what they mean. For example, this proposition says there exists, so it can be proven with a single example.
    \item A picture with dominoes laid out end to end, with 4 in each column, shows this to be true.
    \item At the end of a proof, a small box is often placed at the end to signify the end. This box is \qedsymbol
\end{itemize}

\section{Naming Results}

\section{The Pidgeonhole Principle}

\section{Bonus Examples}

\section{Introduction To Ramsey Theorey}

\end{document}